\documentclass[conference]{IEEEtran}
\IEEEoverridecommandlockouts
% The preceding line is only needed to identify funding in the first footnote. If that is unneeded, please comment it out.
\usepackage[brazilian]{babel}
\usepackage[utf8]{inputenc}
\usepackage[T1]{fontenc}
\usepackage{cite}
\usepackage{amsmath,amssymb,amsfonts}
\usepackage{amsthm}
\usepackage[Algoritmo]{algorithm}
\usepackage{algpseudocode}
\usepackage{graphicx}
\usepackage{textcomp}
\usepackage{xcolor}
\usepackage{hyperref}
\hypersetup{%
    pdfborder = {0 0 0}
}
\def\BibTeX{{\rm B\kern-.05em{\sc i\kern-.025em b}\kern-.08em
    T\kern-.1667em\lower.7ex\hbox{E}\kern-.125emX}}

\newtheorem{obs}{Observa\c{c}\~ao}[section]
\begin{document}

\title{Relatório final MATD74}

\author{\IEEEauthorblockN{Gabriel Dahia}
\IEEEauthorblockA{\textit{Departamento de Ciência da Computação} \\
\textit{Universidade Federal da Bahia}\\
Salvador, Brasil\\
gdahia@gmail.com}
}

\maketitle

\begin{abstract}
  Este relatório t\'ecnico, feito como requisito para disciplina ``MATD74 - Algoritmos e Grafos'' do Programa de Pós-Graduação em Computação da Universidade Federal da Bahia (PGCOMP-UFBA), aborda o problema de construir árvores geradoras com poucos vértices de grau superior a 2.
  Neste documento, definimos formalmente o problema, fazemos uma análise simples de sua categorização na hierarquia de complexidade, mostrando que se trata de um problema NP-Difícil, de difícil aproximação (mas com aproximação trivial do seu problema complementar), revisamos brevemente a literatura sobre o tema, explicamos a heur\'istica proposta, baseada em uma an\'alise utilizando uma abordagem
  de contagem amortizada desenvolvida originalmente para o problema de maximizar o n\'umero de folhas
  em \'arvores geradoras~\cite{solis-oba2017}.
\end{abstract}

\section{Introdução}
A definição formal do problema ao qual nos referiremos como MBST, abreviação de \textit{Minimum Branching-vertices Spannning Tree}, é: dado um grafo $G$, encontrar uma árvore geradora $T$ em que o número de \emph{vértices de ramificação}\footnote{Nossa tradução para \textit{branch vertices}.}, \textit{i.e.} vértices com grau superior a 2, é o menor possivel~\cite{gargano2004}.
Usando $d_T(v)$ para denotar o grau de um vértice $v \in V$ em uma árvore $T$ e $\mathcal{T}$ para denotar o conjunto das árvores geradoras de $G$, o problema pode ser escrito como encontrar $T^*$ tal que:
\begin{equation}
  T^* = \mathrm{arg\,min}_{T \in \mathcal{T}} |\{v \in V \mid d_T(v) > 2 \}| 
\end{equation}

Este problema foi originalmente proposto como uma formulação alternativa ao problema do caminho Hamiltoniano (\textit{Hamiltonian path problem})~\cite{gargano2004}.
A relação entre estes problemas é que um caminho Hamiltoniano é também uma árvore geradora que não possui nenhum vértice de ramificação.
Logo, se um grafo admite um caminho Hamiltoniano, encontrar uma árvore geradora em que o número de vértices de ramificação é mínimo em um grafo é equivalente a encontrar um caminho Hamiltoniano nesse mesmo grafo~\cite{gargano2004}.

Isso por si só é suficiente para estabelecer que MBST é um problema NP-Difícil, já que encontrar um caminho Hamiltoniano em um grafo é um problema NP-Completo~\cite{karp1972}.
Além disso, sabemos que mesmo no caso em que o grafo não admite caminho Hamiltoniano, e portanto possui pelo menos um vértice de ramificação, MBST não pode ser aproximado a uma taxa multiplicativa de $\Omega(\log n)$, em que $n$ é o número de vértices do grafo, da resposta ótima, a não ser que P~=~NP~\cite{salamon2010}.
Consideramos que P $\neq$ NP é uma suposição razoável e a trataremos como verdadeira no que se segue.

A formulação complementar ao problema, \textit{i.e.} encontrar uma árvore geradora que possua o maior número de vértices com grau menor igual a 2, é trivialmente aproximável a um fator multiplicativo de $1/2$ em relação a resposta ótima~\cite{chimani2015}.
Isso ocorre porque, apesar de algoritmos para o problema estarem de fato otimizando a mesma propriedade acerca das árvores geradoras (já que a soma das quantidades desses dois conjuntos de vértices é igual a totalidade de vértices no grafo), para o problema complementar temos um limite superior ao valor do ótimo igual a $n$.
Como toda árvore geradora tem pelo menos $n/2$ vértices com grau 1 ou 2, qualquer algoritmo que encontre uma árvore geradora é um algoritmo aproximativo para esse problema~\cite{chimani2015}.

As seções seguintes trazem um compêndio de trabalhos relacionados e nossa futura metodologia.
A Seção~\ref{sec:review} traz uma breve revisão da literatura e a Seção~\ref{sec:methodology} traz um sumário do que pretendemos investigar.

\section{Revisão da Literatura} \label{sec:review}

A primeira análise do de MBST foi acompanhada de um algoritmo de tempo polinomial para encontrar uma árvore geradora que possui apenas um vértice de ramificação em grafos cuja soma dos graus de qualquer conjunto independente é pelo menos o seu número de vértices subtraido de um~\cite{gargano2004}.

O primeiro, e até onde sabemos único, algoritmo aproximativo para a formulação principal de MBST consiste em construir uma floresta pela escolha de arestas viáveis do vértice com o maior número de vizinhos ainda não selecionados.
Este algoritmo produz no máximo $3 \lceil \log_{1/(1 - c)} n \rceil + 1$ vértices de ramificação, onde $c$ é tal que o número total de arestas do grafo de entrada é igual a $cn$.
Para grafos que não admitam caminho Hamiltoniano, essa é uma aproximação ótima, a menos de uma constante que depende do valor de $c$~\cite{salamon2010}.

Já para a formulação complementar do problema, Chimani e Spoerhase~\cite{chimani2015} foram capazes de determinar que o algoritmo de busca local que otimiza trocas de até 2 arestas de uma árvore geradora encontra soluções com fator de aproximação multiplicativa de $6/11$~\cite{chimani2015}.

A aplicabilidade prática e difícil aproximabilidade do problema motivou pesquisadores a procurarem heurísticas que, apesar de não contarem com garantias teóricas, obtivessem bons resultados na prática, e formulações de programação linear inteira, que, por outro lado, não têm garantias sobre seu tempo de execução, mas costumam ter performance satisfatória.

Cerulli \textit{et al.}~\cite{cerulli2009} foram os primeiros a propor heurísticas para MBST.
Eles propuseram três heurísticas, uma baseada em atribuição de pesos em arestas, uma em um esquema de coloração de vértices, e uma resultante da combinação dessas duas heurísticas.
Estas heurísticas são baseadas em critérios gulosos em relação ao número de vértices de ramificação que seriam criados com a seleção de uma determinada aresta no estado atual de construção da árvore.

Carrabs \textit{et al.}~\cite{carrabs2013} desenvolvem formulações de Programação Inteira e utilizam diferentes relaxações, utilizando instâncias para as quais a solução ótima é conhecida ou as heurísticas propostas anteriormente como comparação para concluir a efetividade dos métodos propostos.

Já Marin~\cite{marin2015} propõe um algoritmo \textit{branch-and-cut} baseado em Programação Inteira capaz de resolver mais instâncias que as abordagens anteriores.
Para as instâncias as quais a formulação não é capaz de resolver, ele propõe o emprego de uma heurística de duas fases.
A primeira fase dessa heurística é baseada no algoritmo de Prim para árvores geradoras mínimas~\cite{prim1957}, enquanto a segunda utiliza otimizações locais~\cite{marin2015}.

Melo \textit{et al.}~\cite{melo2016} propõem técnicas de decomposição para reduzir o problema a tamanhos para os quais soluções de Programação Inteira são tratáveis.
Essas decomposições envolvem determinar vértices que obrigatoriamente são de ramificação em qualquer árvore geradora ou remover arestas de corte do grafo e resolver o problema para os subgrafos resultantes.
Além disso, os autores também detalham uma heuristica baseada na construção gulosa de caminhos longos em duas etapas, efetivamente minimizando o número de ramificações~\cite{melo2016}.

\section{Metodologia} \label{sec:methodology}

A metodologia que nos propomos para o trabalho final dessa disciplina é, até onde sabemos, inédita para MBST.
Nosso trabalho \'e baseado na abordagem de Solis-Oba \textit{et al.} para aproximação de árvores
geradoras com muitas folhas~\cite{solis-oba2017}, e visa nortear a formulação de uma heurística
analisando o problema complementar à MBST.

Essa abordagem, uma t\'ecnica de contagem amortizada de uma diferen\c{c}a, consiste em definir uma
quantia, que os autores chamam de \emph{potencial}~\cite{solis-oba2017}, e desenvolver um algoritmo
que fa\c{c}a com que a varia\c{c}\~ao de potencial entre duas itera\c{c}\~oes do algoritmo seja
sempre n\~ao-negativa. Como tamb\'em temos a garantia de que o potencial \'e n\~ao-negativo antes
da inicializa\c{c}\~ao do algoritmo, ao final de sua execu\c{c}\~ao, temos, por indu\c{c}\~ao, que
o potencial \'e tamb\'em n\~ao-negativo.

Informalmente, definindo o potencial como a diferen\c{c}a entre uma medida de qualidade da nossa
solu\c{c}\~ao e um valor superior a solu\c{c}\~ao \'otima, essa t\'ecnica permite provar taxas de
aproxima\c{c}\~ao para algoritmos. Na pr\'oxima se\c{c}\~ao, definimos formalmente essa abordagem.

\subsection{Potencial de uma \'arvore geradora} \label{sec:potential}
Seja $P(T)$ o n\'umero de v\'ertices com grau menor ou igual a 2 na \'arvore $T$, \textit{i.e.} o
n\'umero de v\'ertices de $T$ que n\~ao s\~ao ramifica\c{c}\~oes. Seja $V(T)$ o n\'umero total de
v\'ertices de $T$.

Considere uma sequ\^encia de \'arvores $T_0, \dots, T_n$ constru\'idas durante as itera\c{c}\~oes
de um algoritmo, onde $T_0$ \'e a \'arvore vazia e $T_k$ \'e uma \'arvore geradora do grafo de
entrada, e definimos $T_j = T_k, j > k$ por conveni\^encia.

O \emph{potencial} $\mathcal{P}(T; x)$ de uma \'arvore \'e definido como:
\begin{equation} \label{eq:potential}
  \mathcal{P}(T; x) = xP(T) - V(T).
\end{equation}
Quando claro ou irrelevante para o que estivermos dizendo, omitiremos a depend\^encia do param\^etro
$x$.

Queremos provar, para um determinado algoritmo, que $\mathcal{P}(T_n) \ge 0$. Com isso, sabemos que
\begin{equation}
   xP(T_n) - V(T_n) \ge 0
\end{equation}
e, portanto,
\begin{equation} \label{eq:tigher-bound}
  P(T_n) \ge V(T_n)/x.
\end{equation}
Como sabemos que $V(T_n) = n \ge P(T^*)$, temos
\begin{equation} \label{eq:pot-obj}
  P(T_n) \ge P(T^*)/x,
\end{equation}
e o algoritmo para o qual provamos esse fato \'e uma aproxima\c{c}\~ao de fator $x$ para o problema
complementar a MBST.

Para provar que $\mathcal{P}(T_n) \ge 0$, a abordagem de
Solis-Oba~\textit{et al.}~\cite{solis-oba2017} \'e provar que $\mathcal{P}(T_0) \ge 0$ e que para
todo $i$, a diferen\c{c}a entre os potencias de $i$ e $i+1$
\begin{equation}
  \Delta \mathcal{P}_{i + 1} = \mathcal{P}(T_{i + 1}) - \mathcal{P}(T_i)
\end{equation}
tamb\'em \'e n\~ao negativa, \textit{i.e.} $\Delta \mathcal{P}_{i + 1} \ge 0, \forall i$.

\subsection{Demonstra\c{c}\~ao de aproximabilidade trivial usando BFS}

Aqui, n\'os usamos o que foi definido na Se\c{c}\~ao~\ref{sec:potential} para detalhar uma
demonstra\c{c}\~ao alternativa de que todo grafo admite uma \'arvore geradora em que pelo menos
metade de seus v\'ertices n\~ao s\~ao bifurca\c{c}\~oes~\cite{chimani2015}.
Na realidade, esse resultado \'e muito mais expressivo do que o que provamos aqui: \emph{qualquer}
\'arvore geradora satisfaz essa propriedade~\cite{chimani2015}.

\begin{obs}
  Pelo menos metade dos v\'ertices de \'arvores geradoras obtidas com o algoritmo de busca em largura
  n\~ao s\~ao ramifica\c{c}\~oes.
\end{obs}
\begin{proof}
  $T_0$, a \'arvore vazia, anterior ao come\c{c}o da BFS, trivialmente satisfaz
  $\mathcal{P}(T_0) \ge 0$ (ambas quantias s\~ao 0). O primeiro passo do algoritmo \'e igualmente
  simples, pois sabemos que o grau m\'inimo do grafo \'e 3 (caso contr\'ario, n\~ao ter\'iamos
  nenhuma ramifica\c{c}\~ao) e podemos escolher qualquer v\'ertice como raiz da \'arvore.
  Essa an\'alise \'e similar ao segundo caso analisado em seguida e segue a desigualdade
  (\ref{eq:essential}).

  No $i$-\'esimo passo do algoritmo, adicionamos os $q$ vizinhos n\~ao visitados de um v\'ertice
  \`a \'arvore $T_i$ (uma opera\c{c}\~ao \`a qual partir de agora nos referiremos como
  ``expans\~ao do v\'ertice), obtendo a \'arvore $T_{i + 1}$. Por praticidade, dividimos a
  an\'alise de $\Delta \mathcal{P}_{i + 1}$ em dois casos, $q = 1$ e $q \ge 2$.
  \begin{itemize}
    \item Para o caso em que $q = 1$, temos que ambos $P(T)$ e $V(T)$ crescem em uma unidade. Da\i
      \begin{equation}
        \Delta \mathcal{P}_{i + 1} = x - 1 \ge 0
      \end{equation}
      e, portanto, precisamos que
      \begin{equation} \label{eq:x-bound}
        x \ge 1
      \end{equation}
      para satisfazermos a desigualdade (\ref{eq:pot-obj}) e obtermos a aproximabilidade de fator
      $x$. Note, agora, que desigualdades como (\ref{eq:x-bound}) nos d\~ao um limite inferior para
      $x$ e que n\'os desejamos \emph{minimizar o seu valor}, pois quanto menor, melhor a taxa de
      aproxima\c{c}\~ao. No caso espec\'ifico da desigualdade (\ref{eq:x-bound}), essa \'e a melhor
      taxa poss\'ivel: usando a desigualdade (\ref{eq:tigher-bound}), obtemos que nessa \'arvore
      n\~ao h\'a ramifica\c{c}\~oes. Isso est\'a de acordo com a expectativa que tinh\'amos ao
      analisar o caso em que o n\'umero de vizinhos $q = 1$, pois se todas as expans\~oes forem desse
      tipo, estaremos construindo um Caminho Hamiltoniano.
    \item Para o caso em que $q \ge 2$, temos que $P(T)$ cresce em $q$, mas perde uma unidade, j\'a
      que o v\'ertice expandido passa a ser uma ramifica\c{c}\~ao. A varia\c{c}\~ao para $P(T)$,
      ent\~ao, \'e de $q - 1$. $V(T)$, por outro lado, \'e acrescido de $q$. Substituindo esses
      valores, temos que
      \begin{equation}
        \Delta \mathcal{P}_{i + 1} = x(q - 1) - q \ge 0.
      \end{equation}
      Reorganizando os termos, obtemos
      \begin{equation} \label{eq:essential}
        x \ge q/(q - 1).
      \end{equation}
      No pior caso, $q = 2$ e obtemos que $x \ge 2$.
  \end{itemize}
  Logo, pela an\'alise dos dois casos, sabemos que \'e suficiente substituir $x = 2$ na
  deriva\c{c}\~ao da Se\c{c}\~ao~\ref{sec:potential} para obter a aproxima\c{c}\~ao de fator 2 para
  as \'arvores geradoras obtidas com busca em largura.
\end{proof}

\subsection{Heur\'istica local baseada em \textit{insight} te\'orico}
A desigualdade (\ref{eq:essential}) nos diz que quanto maior o n\'umero de vizinhos $q$, menor o
valor de $x$ pode se tornar, melhorando a taxa de aproxima\c{c}\~ao do algoritmo.

Tamb\'em podemos concluir, da an\'alise anterior, que caso n\~ao precis\'assemos expandir v\'ertices
com dois vizinhos, poder\'iamos ter uma taxa de aproxima\c{c}\~ao $x = 3/2$.

Com essa observa\c{c}\~ao em mente, propomos a heur\'istica (\ref{algo:heuristic}) para tentar evitar a expans\~ao de v\'ertices com 2 vizinhos n\~ao expandidos. Esse algoritmo \'e essencialmente um
algoritmo que gulosamente expande os v\'ertices de maior grau na vizinha\c{c}a da \'arvore sendo
gerada, com exe\c{c}\~ao do \textbf{if} da linha \ref{algo:if}, que evita a escolha problem\'atica de
v\'ertices com 2 vizinhos.

A sua complexidade \'e $O(n^2)$, j\'a que a cada uma das $O(n)$ itera\c{c}\~oes do \textbf{while} da linha \ref{algo:while} precisamos inspecionar o n\'umero de vizinhos n\~ao expandido de todos os v\'ertices da fronteira da \'arvore em $O(n)$. Existe uma depend\^encia linear no n\'umero de arestas do
grafo, mas ela \'e dominada pelo termo $O(n^2)$. N\~ao h\'a um termo $O(nm)$ na complexidade do
algoritmo porque precisamos atualizar no m\'aximo $O(n)$ arestas no total para formar uma \'arvore
geradora.

\begin{algorithm}
  \caption{Algoritmo heur\'istico para evitar expans\~ao de v\'ertices com 2 vizinhos n\~ao expandidos.}
  \label{algo:heuristic}
    \begin{algorithmic}[1] % The number tells where the line numbering should start
        \Procedure{HeuristicaLocal}{$G$}
            \State $T \gets \emptyset$
            \State $v \gets$ v\'ertice de maior grau em $G$
            \State $T \gets \mathrm{expanda}(T, v)$
            \State
            \While{$T$ n\~ao for geradora} \label{algo:while}
              \State $u \gets$ v\'ertice de $T$ com mais vizinhos fora de $T$
              \If{$\mathrm{deg}_T(u) = 2$ e $\exists w, \mathrm{deg}_T(w) = 1$} \label{algo:if}
                \State $u \gets$ $w$ \Comment{Evita expandir $u$ com 2 vizinhos}
              \EndIf
              \State $T \gets \mathrm{expanda}(T, u)$
            \EndWhile
            \State
            \State \textbf{return} $T$
        \EndProcedure
    \end{algorithmic}
\end{algorithm}

\section{Resultados}
Nesta se\c{c}\~ao, avaliamos o desempenho da heur\'istica proposta de acordo com as inst\^ancias
disponibilizadas por
Carrabs~\textit{et~al.}~\cite{carrabs2013}\footnote{\url{http://www.dipmat2.unisa.it/people/carrabs/www/DataSet/MBV\_Instances.zip}}.
N\'os comparamos a heur\'istica proposta com resultados de duas outras heur\'isticas anteriores
que tamb\'em tinham resultados reportados para essas mesmas
inst\^ancias~\cite{carrabs2013, cerulli2009, melo2016}.

Para obter os resultados de cada inst\^ancia individualmente, recorremos ao ao c\'odigo fonte das
tabelas de um trabalho anterior~\cite{melo2016}, obtidos atrav\'es da submiss\~ao do mesmo ao
\textit{arXiv}\footnote{https://arxiv.org/e-print/1509.06562}. Parseamos os resultados das tabelas
usando a biblioteca Pandas~\cite{pandas}.

Para sermos mais sucintos, apenas reportamos a m\'edia da diferen\c{c}a entre o n\'umero de ramifica\c{c}\~oes
encontrado pelos m\'etodos anteriores e o n\'umero de ramifica\c{c}\~oes encontrado pela heur\'istica proposta.
Os resultados obtidos podem ser encontrados na Tabela~\ref{table:results} e s\~ao restritos \`as 175
inst\^ancias para as quais encontramos os resultados nos trabalhos anteriores.

Estes resultados, que devem ser interpretados como ``a heur\'istica proposta ger
\'arvores com $y$ ramifica\c{c}\~oes a mais que a t\'ecnica anterior A, em m\'edia'',
mostram que existem trabalhos muito mais eficazes na literatura. Interessantemente, o m\'etodo
que se sai melhor na compara\c{c}\~ao com nosso resultado, otimiza de maneira indireta o objetivo
definido pelo potencial da equa\c{c}\~ao (\ref{eq:potential}), pois constroi caminhos longos.

\begin{table}
  \centering
  \caption{Resultados comparativos entre heur\'istica proposta e m\'etodos reportados na literatura.
  Note que os m\'etodos reportados pelos trabalhos anteriores s\~ao o melhor resultado obtido para
  um grupo de heur\'isticas, n\~ao apenas uma t\'ecnica. Ainda assim, o resultado da heur\'istica
  proposta deixa a desejar.}
  \begin{tabular}{|c|c|c|}
    \hline
             & Melo \textit{et al.}~\cite{melo2016} & Carrabs \textit{et al.}~\cite{carrabs2013} \\
    \hline
    Proposta & 17.76 & 15.95 \\
    \hline
  \end{tabular}
  \label{table:results}
\end{table}

\section{Conclus\~ao e trabalhos futuros}
Definir um potencial mais sofisticado, como feito por Solis-Oba \textit{et al.}~\cite{solis-oba2017}, e garantir que alguns dos v\'ertices de grau 2 s\~ao obrigatoriamente ramifica\c{c}\~oes pode ser um caminho para garantir a taxa de aproxima\c{c}\~ao que nossa heur\'istica tenta obter sem garantias.

\bibliographystyle{IEEEtran}
\bibliography{biblio}

\end{document}
