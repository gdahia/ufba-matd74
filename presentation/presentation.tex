\documentclass[14pt]{beamer}

\usepackage[utf8]{inputenc}

%Information to be included in the title page:
\title{\'Arvores geradoras com poucas ramifica\c{c}\~oes}
\subtitle{Trabalho pr\'atico MATD74}
\author{Gabriel Dahia}
\institute{Universidade Federal da Bahia}
\date{2019}

\begin{document}

\frame{\titlepage}

\begin{frame}
\frametitle{Defini\c{c}\~oes}
\begin{itemize}
\item<1-> $G = (V, E)$: simples, conexo, n\~ao-direcionado, $n$ v\'ertices, $m$ arestas;
\item<2-> $T \in \mathcal{T}(G)$: \'arvore geradora de $G$;
\item<3-> \emph{Ramifica\c{c}\~ao}: v\'ertice com grau superior a 2;
\item<4-> $b(T)$: n\'umero de ramifica\c{c}\~oes em $T$;
\item<5-> $P(T)$: n\'umero de n\~ao-ramifica\c{c}\~oes em $T$.
\end{itemize}
\end{frame}

\begin{frame}
\frametitle{Problema MBST}
\begin{block}{\textit{Minumum Branches Spanning Tree} (MBST)}
Dado $G = (V, E)$, queremos $T^*$ tal que
\begin{equation}
T^* = \mathrm{arg\,min}_{T \in \mathcal{T}(G)} b(T)
\end{equation}
\end{block}
\end{frame}

\begin{frame}
\frametitle{Problema MBST}
\begin{figure}
\includegraphics[width=0.7\textwidth]{figures/minbst1.png}
\end{figure}
\end{frame}

\begin{frame}
\frametitle{Problema MBST}
\begin{figure}
\includegraphics[width=0.7\textwidth]{figures/minbst2.png}
\end{figure}
\end{frame}

\begin{frame}
\frametitle{Problema MBST}
\begin{figure}
\includegraphics[width=0.7\textwidth]{figures/minbst3.png}
\end{figure}
\end{frame}

\begin{frame}
\frametitle{Problema MBST}
\begin{figure}
\includegraphics[width=0.7\textwidth]{figures/minbst4.png}
\end{figure}
\end{frame}

\begin{frame}
\frametitle{Aplica\c{c}\~oes}
\begin{itemize}
\item<1-> Vers\~ao do \emph{Caminho Hamiltoniano};
\item<2->Redes \'opticas:
\begin{itemize}
  \item Comunica\c{c}\~ao \textit{multicast};
  \item Multiplexa\c{c}\~ao (DWDM).
\end{itemize}
\end{itemize}
\end{frame}

\begin{frame}
\frametitle{Problemas relacionados}
\begin{itemize}
\item<1->Problema NP-D\'ificil;
\item<2->Caminho Hamiltoniano;
\item<3->Minimizar folhas – maximizar v\'ertices internos;
\item<4->Maximizar v\'ertices n\~ao-ramificantes.
\end{itemize}
\end{frame}

\begin{frame}
\frametitle{T\'ecnicas anteriores}
\begin{itemize}
\item<1->Algoritmos exatos:
\begin{itemize}
\item polinomiais p/ casos espec\'ificos;
\item aproximativos.
\end{itemize}
\item<2->Programa\c{c}\~ao inteira;
\item<3->Heur\'isticas.
\end{itemize}
\end{frame}

\begin{frame}
\frametitle{Heur\'isticas conhecidas}
\begin{itemize}
% ambos sao objetivos validos: se conseguirmos otimizar essas
% quantidades perfeitamente, nos otimizamos MBST. (explicar como)
\item Algoritmos gulosos:
\begin{itemize}
\item arestas que criem menos ramifica\c{c}\~oes;
\item caminhos longos.
\end{itemize}
\end{itemize}
\end{frame}

\begin{frame}
\frametitle{Potenciais e an\'alise amortizada}
\begin{itemize}
\item<1-> Proposto por Solis-Oba \textit{et al.} (1998; 2015);
\item<2-> Usada para aproximar maximiza\c{c}\~ao de folhas;
\item<3-> Contagem amortizada de propor\c{c}\~oes.
\end{itemize}
\end{frame}

\begin{frame}
\frametitle{Potenciais e an\'alise amortizada}
Defina
\begin{equation}
\mathcal{P}(T) = xP(T) - V(T).
\end{equation}
\end{frame}

\begin{frame}
\frametitle{Potenciais e an\'alise amortizada}
Queremos provar que
\begin{equation}
\mathcal{P}_n \ge 0
\end{equation}
mostrando que
\begin{equation}
\mathcal{P}_0 \ge 0
\end{equation}
e
\begin{equation}
\Delta \mathcal{P}_{i + 1} = \mathcal{P}(T_{i + 1}) - \mathcal{P}(T_i) \ge 0, \forall i
\end{equation}
\end{frame}

\begin{frame}
\frametitle{Potenciais e an\'alise amortizada}
Fazendo isso, temos
\begin{equation}
\mathcal{P}_n = xP(T_n) - n \ge 0
\end{equation}
e, portanto,
\begin{equation}
P(T_n) \ge \frac{n}{x} \ge \frac{P(T^*)}{x}
\end{equation}
\end{frame}

\begin{frame}
\frametitle{Heur\'istica local}
\begin{block}{Ideia}
N\~ao expandir (adicionar na \'arvore em constru\c{c}\~ao) $v$ com 2 vizinhos \'e algoritmo aprox.
\end{block}
\pause
\begin{block}{Por qu\^e?}
Expandir $v$ com $k \ge 2$ vizinhos faz com que
\begin{itemize}
\item $P(T)$ aumente em $k - 1$;
\item $V(T)$ aumente em $k$;
\end{itemize}
\end{block}
\end{frame}

\begin{frame}
\frametitle{Heur\'istica local}
\begin{block}{Por qu\^e?}
Logo,
\begin{equation}
\Delta \mathcal{P}_{i + 1} \ge x(k - 1) - k \ge 0
\end{equation}
\begin{equation}
x \ge k/(k - 1).
\end{equation}
\pause
Se $k = 2$,
\begin{equation}
x \ge 2.
\end{equation}
\end{block}
\end{frame}

\begin{frame}
\frametitle{Demonstra\c{c}\~ao de aproxima\c{c}\~ao trivial}
\begin{block}{Teorema (Chimani \& Spoerhase, 2015)}
Qualquer grafo admite 2-aprox.
\end{block}

\begin{block}{Demonstra\c{c}\~ao}<2->
Construa a \'arvore usando BFS e fa\c{c}a $x = 2$.
\end{block}
\end{frame}

\begin{frame}
\frametitle{Heur\'istica local}
\begin{block}{Algoritmo}
Expanda o v\'ertice que
\begin{enumerate}
  \item possui apenas um vizinho n\~ao expandido; ou
  \item possui o maior n\'umero de vizinhos, caso n\~ao haja com apenas um,
\end{enumerate}
at\'e que a \'arvore seja geradora.
\end{block}
\pause
Complexidade: $O(V^2)$.
\end{frame}

\begin{frame}
\frametitle{Conclus\~oes e poss\'iveis extens\~oes}
\begin{itemize}
\item Proposta de heur\'istica usando \textit{insight} te\'orico;
\item<2-> Objetivo para m\'etodos de aprendizado;
\item<3-> Considerar classes espec\'ificas de grafos;
\item<4-> Tratar casos como em Solis-Oba \textit{et al.} (2015).
\end{itemize}
\end{frame}

\begin{frame}
\begin{center}
\huge
Obrigado!
\end{center}
\end{frame}

\end{document}
